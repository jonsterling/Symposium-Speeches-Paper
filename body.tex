The six speeches of the \emph{Symposium} may be divided into two sets:
on the one hand, there is Phaedrus's speech and those which are
derivative of his, and on the other hand, there are the novel speeches;
and these are the ones which reject the previous approaches and propose
unique theories of Eros.

\subsection*{Phaedrus and his derivatives}

Phaedrus's speech may be understood as the basis for a theory of Eros
which is developed and augmented all the way through the speeches of
Pausanias and Eryximachus.

If we put aside Phaedrus's initial mythological appeals---which I
consider to be a side effect of the kind of speech he is giving, and
much less a core part of his argument---the main focus of Phaedrus's
speech, then, would seem to be the power of Eros to engender in humans a
kind of virtue or fury which they might not have had without the god's
inspiration.

When Pausanias begins, he criticizes Phaedrus for failing to include in
his account the fact that Eros is double in nature:

\begin{quote}
\textgreek{\hlig{εἰπεῖν} δ' αὐτὸν ὅτι Οὐ καλῶς μοι \hlig{δοκεῖ}, ὦ
Φαῖδρε, προβεβλῆσ\salt{θ}αι ἡμῖν ὁ λόγος, τὸ ἁπλῶς οὕτως
παρηγγέλ\salt{θ}αι \hlig{ἐγκωμιάζειν} Ἔρωτα. εἰ μὲν γὰρ \hlig{εἷς} ἦν
ὁ Ἔρως, καλῶς ἂν \hlig{εἶχε}, νῦν δὲ οὐ γάρ ἐστιν \hlig{εἷς}· μὴ ὄντος
δὲ ἑνός ὀρ\salt{θ}ότερόν ἐστι πρότερον προρη\salt{θ}ῆναι ὁποῖον
\hlig{δεῖ} \hlig{ἐπαινεῖν}.} (180c3--180d1)\footnote{And he said, ``It
doesn't seem right to me, O Phaedrus, for the speech to be proposed,
that is, for it to be exhorted to just simply praise Eros. For if Eros
were indeed singular, then it would be fine; but in fact, he is not
singular; and since he is not singular, then it would more correct to
say beforehand in what fashion one ought to praise him.''}
\end{quote}

But Pausanias's analysis is necessarily a refinement of Phaedrus's, and
not a refutation. In fact, Pausanias only adds and does not subtract
from Phaedrus's point; for he simply specifies further both the nature
and the origin of the inspiration that Phaedrus discussed, integrating
it into his bipartite understanding of Eros. In Pausanias's framework,
this inspiration is nothing other than the engenderment of concern for
virtue in the hearts of lovers and their beloveds; and this is, to be
sure, the inspiration which is derived from the Heavenly Eros, as
opposed to the other:

\begin{quote}
\textgreek{οὗτός ἐστιν ὁ τῆς οὐρανίας \salt{θ}εοῦ ἔρως καὶ οὐράνιος
καὶ πο\hlig{λλ}οῦ ἄξιος καὶ \hlig{πόλει} καὶ ἰδιώταις, πο\hlig{λλ}ὴν
\hlig{ἐπιμέλειαν} ἀναγκάζων \hlig{ποιεῖσ\salt{θ}αι} πρὸς ἀρετὴν τόν τε
ἐρῶντα αὐτὸν αὑτοῦ καὶ τὸν ἐρώμενον· οἱ δ' ἕτεροι πάντες τῆς ἑτέρας,
τῆς πανδήμου.} (185b5--185c2)\footnote{This is the love from the
heavenly goddess, and it's heavenly itself, and worthy of much both in
the city and in private, forcing both the lover and the beloved to
have much care for himself in relation to excellence; and the all the
others are of the other (love), the one one of common (Aphrodite).}
\end{quote}

In this way, Pausanias is one of the refiners of previous ideas.
Likewise, Eryximachus fails to present an entirely new approach, but
rather starts where he considers Pausanias to have left off on the way
to the correct theory:

\begin{quote}
\textgreek{\hlig{Εἰπεῖν} δὴ τὸν Ἐρυξίμαχον, \hlig{Δοκεῖ} τοίνυν μοι
ἀναγκαῖον \hlig{εἶναι}, \hlig{ἐπειδὴ} Παυσανίας ὁρμήσας ἐπὶ τὸν λόγον
καλῶς οὐχ ἱκανῶς ἀπετέλεσε, \hlig{δεῖν} ἐμὲ \hlig{πει}ρᾶσ\salt{θ}αι τέλος
\hlig{ἐπι\salt{θ}εῖναι} τῷ λόγῳ.} (185e6--196a2)\footnote{So
Eryximuachus said, ``It seems to me to be necessarily the case, since
Pausanias started into the speech well but failed to finish it
satisfactorily, that I must try and add an end to the speech.''}
\end{quote}

In the course of his analysis, Eryximachus proceeds to generalize the
existing discourse to encompass wide enough a breadth so as to admit a
discussion of Medicine, Music, Athletics, Agriculture, and, in fact,
every other affair.

To Eryximachus, then, the two kinds of love that humans may experience
in their relationships are only special cases of something much more
abstract; that is, Eros is simply the thing which drives opposites to
combine. And so when things at variance combine by means of the Heavenly
Eros (whether they be extremes of humor, flavor, pitch or temperature,
or some other continuum), the result is harmonious and orderly; whereas,
when the differing things combine by means of the Common Eros, the
result is of disorderliness, harm, and pestilence.

By this means, Eryximachus continues with Pausanias's program of
analyzing which kinds of inspiration are engendered by which kind of
Eros, generalizing them to the point where we must understand the
Inspiration introduced by Phaedrus as a special case itself of what
results from, more abstractly, the joining of things together---in this
case, namely, the joining of humans together.

\subsection*{The Novel Speeches}

Aristophanes begins by noting how he intends to pursue a different tack
than that those who preceded him:

\begin{quote}
\textgreek{Καὶ μήν, ὦ Ἐρυξίμαχε, \hlig{εἰπεῖν} τὸν Ἀριστοφάνη,
ἄ\hlig{λλ}ῃ γέ πῃ ἐν νῷ ἔχω \hlig{λέγειν} ἢ  ᾗ σύ τε καὶ Παυσανίας
\hlig{εἰπέτην}.} (189c2--189c3)\footnote{``And yet, O Eryximachus,''
said Aristophanes, ``I have it in mind to speak in a different way, at
least, from how you and Pausanias have spoken.''}
\end{quote}

The approach of Aristophanes is to develop his theory by providing a
mythological backdrop, and then using it to explain the attractions
which occur between humans. At the surface, his argument would almost
appear in the class of explanations which say that different kinds of
love yield different kinds of inspiration, as Pausanias argued. And yet,
I think it can be shown to be not so much that, but rather something a
bit different.

In enumerating the different kinds of love-pursuit that may occur (that
is, men pursuing women, women pursuing men, women pursuing women, and
men pursuing men), Aristophanes distinguishes between different kinds of
love: when men and women pursue each other, this is the spirit of
adultery; when women pursue women, this is the spirit of lesbianism; but
when men pursue men, this is manliness:

\begin{quote}
\textgreek{ὅσοι δὲ ἄρρενος τμῆμά \hlig{εἰσι}, τὰ ἄρρενα διώκουσι, καὶ
τέως μὲν ἂν παῖδες ὦσιν, ἅτε τεμάχια ὄντα τοῦ ἄρρενος, φιλοῦσι τοὺς
ἄνδρας καὶ χαίρουσι \hlig{συγκατακείμενοι} καὶ συμπεπλεγμένοι τοῖς
ἀνδράσι, καί \hlig{εἰσιν} οὗτοι βέλτιστοι τῶν παίδων καὶ μειρακίων,
ἅτε \hlig{ἀνδρειότατοι} ὄντες \hlig{φύσει}.}
(191e1--192a2)\footnote{And all those which are cuttings from the male
pursue men, and so long as they are children, inasmuch as they are
slices of a man, love men and delight in lying with them and embracing
them, and these are the best of children and youths, inasmuch as they
are the most manly by nature.}
\end{quote}

And so, if we wished to characterize Aristophanes's analysis within the
framework of the dual Eros, it would be the male homosexual
relationships which are derived from the Heavenly Eros, and all the
others from the Common one. And yet, such a characterization proves
quite difficult to argue, if we consider the lines of causality between
nature, love and inspiration.

Whereas we right well imagine that Pausanias would consent to the idea
it is a person's nature which determines which kind of Eros will join
him to his beloved (or vice versa), and that it is this in turn which
determines whether or not his inspiration is virtuous, we cannot say the
same of Aristophanes.

For to Aristophanes, the nature of a human (which is to say, whether he
or she derived from a man-man, a woman-woman, or a man-woman) determines
which Love they will experience; this much provokes no disagreement with
the previous discourse, and merely augments it with a comical just-so
story. But it is not \emph{which} Love they experience that inspires
them with civic and virtuous concerns, but rather their nature itself:

\begin{quote}
\textgreek{καὶ γὰρ τελεω\salt{θ}έντες μόνοι ἀποβαίνουσιν \hlig{εἰς} τὰ
πολιτικὰ ἄνδρες οἱ τοιοῦτοι. ἐπειδὰν δὲ ἀνδρω\salt{θ}ῶσι,
παιδεραστοῦσι καὶ πρὸς γάμους καὶ παιδοποιίας οὐ προσέχουσι τὸν νοῦν
\hlig{φύσει}, \hlig{ἀλλ'} ὑπὸ τοῦ νόμου ἀναγκάζονται· ἀ\hlig{λλ'}
\hlig{ἐξαρκεῖ} αὐτοῖς μετ' \hlig{ἀλλ}ήλων καταζῆν ἀγάμοις.}
(192a6--192b3)\footnote{For indeed, when the grow up, only such men
turn out to be politically inclined. And whenever they become men,
they desire boys and do not incline themselves toward marriage and
the making of children by nature, but rather are compelled by custom;
rather, it suffices for them to live out their lives unmarried in
each other's company.}
\end{quote}

It is on account of being derived from a man-man that one is daring and
virile, and it is on account of being that way both that one is inclined
toward homosexual love, and that one is inspired to participate in the
matters of the city. So, the inspiration in Aristophanes's framework
does not derive from the Love, but rather issues directly from the
primitive nature of the man himself, by which the kind of Love to be
experienced is also determined.

Now then, Agathon begins by dismissing all the previous attempts at
explaining Eros as missing the point, by failing to praise the god
himself, but rather only congratulating the humans who are affected by
him:

\begin{quote}
\textgreek{δοκοῦσι γάρ μοι πάντες οἱ πρόσ\salt{θ}εν \hlig{εἰρηκότες}
οὐ τὸν \salt{θ}εὸν \hlig{ἐγκωμιάζειν} \hlig{ἀλλὰ} τοὺς
ἀν\salt{θ}ρώπους \hlig{εὐδαιμονίζειν} τῶν ἀγα\salt{θ}ῶν ὧν ὁ
\salt{θ}εὸς αὐτοῖς αἴτιος· ὁποῖος δέ τις αὐτὸς ὢν ταῦτα ἐδωρήσατο,
οὐδ\hlig{εὶς} \hlig{εἴρηκεν}.} (194e4--195a1)\footnote{For it seems to
me that the ones who have spoken so far were not praising the god, but
rather congratulating the men for the goods of which the god is
responsible for them; but being what kind of god he himself gave these
gifts, nobody has said.}
\end{quote}

The analysis provided by Agathon is, to put it briefly, that Eros
alights upon those who are soft and beautiful, and that he is young and
soft himself. He omits the notion of the dual Eros, and conceives that
he looks for the beautiful and the good, and engenders it in everyone he
touches.

Not to be outdone by Agathon, Socrates starts off by feining to have
misunderstood what an encomium was, having expected to be able to simply
say the most beautiful truths that could be said about the subject so
well as he could, but being instead put to attribute all the greatest
and most beautiful things to the subject, whether or not they are the
truth.

\begin{quote}
\textgreek{ἐγὼ μὲν γὰρ ὑπ' ἐβελτερίας ᾤμην \hlig{δεῖν} τἀλη\salt{θ}ῆ
\hlig{λέγειν} περὶ ἑκάστου τοῦ ἐγκωμιαζομένου, καὶ τοῦτο μὲν
\hlig{ὑπάρχειν}, ἐξ αὐτῶν δὲ τούτων τὰ κά\hlig{λλ}ιστα ἐκλεγομένους ὡς
εὐπρεπέστατα τι\salt{θ}έναι· καὶ πάνυ δὴ μέγα ἐφρόνουν ὡς εὖ ἐρῶν, ὡς
\hlig{εἰδὼς} τὴν \hlig{ἀλή\salt{θ}ειαν} τοῦ \hlig{ἐπαινεῖν} ὁτιοῦν. τὸ
δὲ ἄρα, ὡς ἔοικεν, οὐ τοῦτο ἦν τὸ καλῶς \hlig{ἐπαινεῖν} ὁτιοῦν,
ἀ\hlig{λλ}ὰ τὸ ὡς μέγιστα ἀνατι\salt{θ}έναι τῷ \hlig{πράγματι} καὶ ὡς
κά\hlig{λλ}ιστα, ἐὰν τε ᾖ οὕτως ἔχοντα ἐάν τε μή· εἰ δὲ ψευδῆ, οὐδὲν
ἄρ' ἦν πρᾶγμα.} (198d3--198e2)\footnote{For I in my silliness thought
that one ought to say the truth about each god who is being praised,
and to begin this thing, and to pick out the most beautiful out of
these same things and put them as nicely as possible; and forsooth, I
thought very much indeed that I would speak well, on the grounds that
I knew the truth about praising something. But, as it would turn out,
this is not what is meant by praising something finely, but rather it
is to attribute the greatest things to the subject, and the most
beautiful, whether it be the case or not; and if they are false, it
turns out to be of no account.}
\end{quote}

After some coaxing, Socrates agrees to deliver a praise speech in the
way that he sees fit, distinct from the rest. In this way, Socrates
wipes clean the slate and establishes his framework as one of the novel
ones. Indeed, the theory which he claims to have gotten from Diotima is
more divergent from all the previous speeches than any of them differed
amongst themselves.

\section{Conclusion}

As I have shown, the last three speeches were novel, in that none of
them could be combined with the ones that preceded to form a
non-contradictory theory of Eros. Whilst nobody should be surprised that
Socrates failed to agree with anyone on something, I suggest that in
addition to him, it may be no coincidence at all that it was the poets
of the company, Agathon and Aristophanes, who came up with original and
non-derivative speeches.
