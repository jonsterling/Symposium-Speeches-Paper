The six speeches of the \emph{Symposium} may be divided into two sets:
on the one hand, there is Phaedrus's speech and those which are
derivative of his, and on the other hand, there are the novel speeches;
and these are the ones which reject the previous approaches and propose
unique theories of Eros.

\section{Phaedrus and his derivatives}

Phaedrus's speech may be understood as the basis for a theory of Eros
which is developed and augmented all the way through the speeches of
Pausanias and Eryximachus.

If we put aside for the moment Phaedrus's initial mythological
appeals---which I consider to be a side effect of the kind of speech he
is giving, and much less a core part of his argument---the main focus of
Phaedrus's speech, then, would seem to be the power of Eros to engender
in humans a kind of virtue or fury which they might not have had without
the god's inspiration.

When Pausanias begins, he criticizes Phaedrus for failing to include in
his account the fact that Eros is double in nature:

\begin{quote}
\textgreek{εἰπεῖν δ' αὐτὸν ὅτι Οὐ καλῶς μοι δοκεῖ, ὦ Φαῖδρε,
προβεβλῆσθαι ἡμῖν ὁ λόγος, τὸ ἁπλῶς οὕτως παρηγγέλθαι ἐγκωμιάζειν
Ἔρωτα. εἰ μὲν γὰρ εἷς ἦν ὁ Ἔρως, καλῶς ἂν εἶχε, νῦν δὲ οὐ γάρ ἐστιν
εἷς· μὴ ὄντος δὲ ἑνός ὀρθότερόν ἐστι πρότερον προρηθῆναι ὁποῖον δεῖ
ἐπαινεῖν.} (180c3--180d1)
\end{quote}

But Pausanias's analysis is necessarily a refinement of Phaedrus's, and
not a refutation. In fact, Pausanias only adds and does not subtract
from Phaedrus's point; for he simply specifies further both the nature
and the origin of the inspiration that Phaedrus discussed, integrating
it into his bipartite understanding of Eros. In Pausanias's framework,
this inspiration is nothing other than the engenderment of concern for
virtue in the hearts of lovers and their beloveds; and this is, to be
sure, the inspiration which is derived from the Heavenly Eros, as
opposed to the other:

\begin{quote}
\textgreek{οὗτός ἐστιν ὁ τῆς οὐρανίας θεοῦ ἔρως καὶ οὐράνιος καὶ
πολλοῦ ἄξιος καὶ πόλει καὶ ἰδιώταις, πολλὴν ἐπιμέλειαν ἀναγκάζων
ποιεῖσθαι πρὸς ἀρετὴν τόν τε ἐρῶντα αὐτὸν αὑτοῦ καὶ τὸν ἐρώμενον· οἱ
δ' ἕτεροι πάντες τῆς ἑτέρας, τῆς πανδήμου.} (185b5--185c2)
\end{quote}

In this way, Pausanias is one of the refiners of previous ideas.
Likewise, Eryximachus fails to present an entirely new approach, but
rather starts where he considers Pausanias to have left off on the way
to the correct theory:

\begin{quote}
\textgreek{Εἰπεῖν δὴ τὸν Ἐρυξίμαχον, Δοκεῖ τοίνυν μοι ἀναγκαῖον εἶναι,
ἐπειδὴ Παυσανίας ὁρμήσας ἐπὶ τὸν λόγον καλῶς οὐχ ἱκανῶς ἀπετέλεσε,
δεῖν ἐμὲ πειρᾶσθαι τέλος ἐπιθεῖναι τῷ λόγῳ.} (185e6--196a2)
\end{quote}

In the course of his analysis, Eryximachus proceeds to generalize the
existing discourse to encompass wide enough a breadth so as to admit a
discussion of Medicine, Music, Athletics, Agriculture, and, in fact,
every other affair.

To Eryximachus, then, the two kinds of love that humans may experience
in their relationships are only special cases of something much more
abstract; that is, Eros is simply the thing which drives opposites to
combine. And so when things at variance combine by means of the Heavenly
Eros (whether they be extremes of humor, flavor, pitch or temperature,
or some other continuum), the result is harmonious and orderly; whereas,
when the differing things combine by means of the Common Eros, the
result is of disorderliness, harm, and pestilence.

By this means, Eryximachus continues with Pausanias's program of
analyzing which kinds of inspiration are engendered by which kind of
Eros, generalizing them to the point where we must understand the
Inspiration introduced by Phaedrus as a special case itself of what
results from, more abstractly, the joining of things together---in this
case, namely, the joining of humans to each other.

\section{The Novel Speeches}

Aristophanes begins by noting how he intends to pursue a different tack
than that those how preceded him:

\begin{quote}
\textgreek{Καὶ μήν, ὦ Ἐρυξίμαχε, εἰπεῖν τὸν Ἀριστοφάνη, ἄλλῃ γέ πῃ ἐν
νῷ ἔχω λέγειν ἢ  ᾗ σύ τε καὶ Παυσανίας εἰπέτην.} (189c2--189c3)
\end{quote}

The approach of Aristophanes is to develop his theory by providing a
mythological backdrop, and then using it to explain the attractions
which occur between humans. At the surface, his argument would almost
appear in the class of explanations which say that different kinds of
love yield different kinds of inspiration, as Pausanias argued. And yet,
I think it can be shown to be not so much that, but rather its converse.

In enumerating the different kinds of love-pursuit that may occur (that
is, men pursuing women, women pursuing men, women pursuing women, and
men pursuing men), Aristophanes distinguishes between different kinds of
love: when men and women pursue each other, this is the spirit of
adultery; when women pursue women, this is the spirit of lesbianism; but
when men pursue men, this is manliness, courage and boldness:

\begin{quote}
\textgreek{φασὶ δὲ δή τινες αὐτοὺς ἀναισχύντους εἶναι, ψευδόμενοι· οὐ
γὰρ ὑπ' ἀναισχυντίας τοῦτο δρῶσιν ἀλλ' ὑπὸ θάρρους καὶ ἀνδρείας καὶ
ἀρρενωπίας, τὸ ὅμοιον ἀυτοῖς ἀσπαζόμενοι.} (192a2--192a5)
\end{quote}

And so, if we wished to characterize Aristophanes's analysis within the
framework of the dual Eros, it would be the male homosexual
relationships which are derived from the Heavenly Eros, and all the
others from the Common one. And yet, such a characterization proves
quite difficult to argue, if we consider the lines of causality between
nature, love and inspiration.

Whereas we right well imagine that Pausanias would consent to the idea
it is a person's nature which determines which kind of Eros will join
him to his beloved (or vice versa), and that it is this in turn which
determines whether or not his inspiration is virtuous and orderly, we
cannot say the same of Aristophanes.

For to Aristophanes, the nature of a human (which is to say, whether he
or she derived from a man-man, a woman-woman, or a man-woman) determines
which Love they will experience; this much provokes no disagreement with
the previous discourse, and merely augments it with comical just-so
story. But it is not \emph{which} Love they experience that inspires
them with civic and orderly concern, but rather their nature itself:

\begin{quote}
\textgreek{καὶ γὰρ τελεωθέντες μόνοι ἀποβαίνουσιν εἰς τὰ πολιτικὰ ἄνδρες οἱ
τοιοῦτοι. ἐπειδὰν δὲ ἀνδρωθῶσι, παιδεραστοῦσι καὶ πρὸς γάμους καὶ
παιδοποιίας οὐ προσέχουσι τὸν νοῦν φύσει, ἀλλ' ὑπὸ τοῦ νόμου
ἀναγκάζονται· ἀλλ' ἐξαρκεῖ αὐτοῖς μετ' ἀλλήλων καταζῆν ἀγάμοις.}
(192a6--192b3)
\end{quote}

It is on account of being derived from a man-man that one is daring and
virile, and it is on account of being that way both that one is inclined
toward homosexual love, and that one is inspired to participate in the
matters of the city. So, the inspiration in Aristophanes's framework
does not derive from the Love, but rather issues directly from the
primitive nature of the man himself, by which the kind of Love to be
experienced is also determined.

Now then, Agathon begins by dismissing all the previous attempts at
explaining Eros as missing the point, by failing to praise the god
himself, but rather only congratulating the humans who are affected by
him:

\begin{quote}
\textgreek{δοκοῦσι γάρ μοι πάντες οἱ πρόσθεν εἰρηκότες οὐ τὸν θεὸν
ἐγκωμιάζειν ἀλλὰ τοὺς ἀνθρώπους εὐδαιμονίζειν τῶν ἀγαθῶν ὧν ὁ θεὸς
αὐτοῖς αἴτιος· ὁποῖος δέ τις αὐτὸς ὢν ταῦτα ἐδωρήσατο, οὐδεὶς
εἴρηκεν.} (194e4--195a1)
\end{quote}

The analysis provided by Agathon is, to put it briefly, that Eros
alights upon those that are soft and beautiful, and is young and soft
himself. He omits the notion of the dual Love, and conceives that Love
looks for the beautiful and the good, and engenders it in everything he
touches.

Not to be outdone by Agathon, Socrates starts off by feining to have
misunderstood what an encomium was, having expected to be able to simply
say the most beautiful truths that could be said about the subject so
well as he could, but being instead put to attribute all the greatest
and most beautiful things to the subject, whether or not they are the
truth.

\begin{quote}
\textgreek{ἐγὼ μὲν γὰρ ὑπ' ἐβελτερίας ᾤμην δεῖν τἀληθῆ λέγειν περὶ
ἑκάστου τοῦ ἐγκωμιαζομένου, καὶ τοῦτο μὲν ὑπάρχειν, ἐξ αὐτῶν δὲ τούτων
τὰ κάλλιστα ἐκλεγομένους ὡς εὐπρεπέστατα τιθέναι· καὶ πάνυ δὴ μέγα
ἐφρόνουν ὡς εὖ ἐρῶν, ὡς εἰδὼς τὴν ἀλήθειαν τοῦ ἐπαινεῖν ὁτιοῦν. τὸ δὲ
ἄρα, ὡς ἔοικεν, οὐ τοῦτο ἦν τὸ καλῶςσ ἐπαινεῖν ὁτιοῦν, ἀλλὰ τὸ ὡς
μέγιστα ἀνατιθέναι τῷ πράγματι καὶ ὡς κάλλιστα, ἐὰν τε ᾖ οὕτος ἔχοντα
ἐάν τε μή· εἰ δὲ ψευδῆ, οὐδὲν ἄρ' ἦν πρᾶγμα.} (198d3--198e2)
\end{quote}

After some coaxing, Socrates agrees to deliver a praise speech in the
way that he sees fit, distinct from the rest. In this way, Socrates
wipes clean the slate and establishes his framework as one of the novel
ones. Indeed, the theory which he claims to have gotten from Diotima is
more divergent from all of the previous speeches than any of them
differed amongst themselves.

\section{Conclusion}

As I have shown, the last three speeches were novel, in that none of
them could be combined with the ones that preceded to form a
non-contradictory theory of Eros. Whilst nobody should be surprised that
Socrates failed to agree with anyone on something, I suggest that in
addition to him, it may be no coincidence at all that it was the poets
of the company, Agathon and Aristophanes, who came up with the most
original and non-derivative speeches.
